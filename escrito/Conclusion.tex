\section{Conclusion and Future Work}

The NeuroGrubi was the first step towards to develop a web tool for 
neural networks. It certainly proved to be feasible by enabling people and 
organizations to have access a simple and free neural networks tool.

The research also shows that the technology of artificial neural networks can 
easily reach users not experts, since the developers have in mind what kind of 
user applications are intended to facilitate the use, leaving the fewest 
accessible possible to the configuration parameters. The NeuroGrubi was 
implemented with the goal of being easy to  non-experts users.

It is known that the arise of a new idea, as well as research on its 
feasibility and first steps for its implementation are quite difficult and 
complicated. It is believed that this step was completed this work, however, 
there is no doubt that there is much more to be done: starting with the 
implementation and analysis of other models of Artificial Neural Networks. 
Perhaps the development of a script specifying the steps to be carried out was 
of vital importance . Another possible work is changing the server-side 
technology, because \em{nodejs} and \em{javascript} are easy handling but are 
shown not to have a good performance when the script to train was running. 
Changing the programming language to \em{C++} or \em{Java} can improve the use 
of computer resources .

It is also essential to develop other applications more elaborate and refined. 
Could be carried out a research on various companies to find out which areas 
of Artificial Neural Networks techniques could be applied to improve 
performance and/or lower costs. Thus, the study could generate, with low cost, 
easily accepted tools and understanding to those potentially interested in 
a future purchase.

It is true that in one way or another, most applications and all the attention 
will be focused here on to the Internet. Offering these prototypes as web 
application will not only facilitate the exchange of idea on the part of 
developers, researchers and students as well. The internet makes the NeuroGrubi 
easier to access for people interested in using and learning Artificial Neural 
Network.