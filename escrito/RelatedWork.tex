\section{Related Work}

The technology of artificial neural networks has been widely used in recent 
years in several areas of application. The use of this technology is already 
widespread in many areas, where the great performance objectives, monitoring and 
systems integration activities are targeted and increasingly arise solutions 
that use neural networks in new areas that exceed the solutions obtained by 
conventional \cite{1327478}.

Examples of some areas where the use of artificial neural networks have achieved 
great success:

\begin{itemize}
\item Water Flow Forecasting \cite{1556318}
\item Prediction of Electricity demand \cite{4679182}
\item Character Recognition \cite{5726693}
\item Stock Market Predictor \cite{5715226}
\item Signature Validation \cite{549040}
\item Characterizing and Evaluating Fraud in Electronic Transactions 
\cite{6392147}
\item Medical Image Diagnosis \cite{5953960}
\end{itemize}

For access to neural networks help a larger group of people, this work suggested 
the development of a web application that provides the user with an interesting 
solution to their problems.

It is apparent that a neural network has its computational power primarily 
related to its structure highly distributed and parallel and also their ability 
to learn and therefore generalize (ability of the neural network has to 
produce reasonable outputs for inputs not shown in phase learning/training). 
These two information-processing capabilities make it possible to solve complex 
problems through the use of techniques of Artificial Neural Networks. In 
practice, however, artificial neural networks can not by itself generate 
solutions. Should rather be integrated into a consistent approach to Systems 
Engineering. More specifically, one should take the complex problem which we 
wish to arrive at a solution, break it down into a number of relatively simple 
tasks and assign certain of these tasks (pattern recognition, associative 
memory, control, etc.). Techniques Artificial Neural Networks to solve them.

The use of Artificial Neural Networks offers the following useful properties and 
capabilities \cite{Bishop2005}:

\begin{enumerate}[I]
\item Non-Linearity.
\item Input-output Mapping.
\item Adaptation.
\item Contextual Information.
\item Failure tolerance.
\end{enumerate}