\documentclass[10pt]{article}
\usepackage[utf8]{inputenc}
\usepackage{infocomp}
\usepackage{times}
\usepackage{amsmath}
\usepackage{amssymb}
\usepackage[T1]{fontenc}
%\usepackage[brazilian]{babel}
\usepackage{graphicx}
\usepackage{hyperref}
\usepackage{subfigure}
\usepackage{enumerate}
\usepackage{caption}
\sloppy
\renewcommand{\captionfont}{\footnotesize}
\renewcommand{\captionlabelfont}{\footnotesize \bfseries}

\newtheorem{exemplo}{Exemplo}

\address{UFLA - Universidade Federal de Lavras\\
        DCC - Departmento de Ci\^encia da Computa\c{c}\~ao\\
        P.O. Box 3037 - Campus da UFLA
		37200-000 - Lavras (MG)- Brazil\\
        $^1$\url{luizcosta@posgrad.ufla.br}}

\title{NeuroGrubi: a Web Application of Artificial Neural Networks Simulator}

\author{
        Luiz Augusto Guimarães Costa$^1$ 
}

\abstract{This paper aims to develop a methodology that facilitates the use 
of Artificial Neural Networks and more broadly to initiate a process of using 
the revolutionary techniques of Artificial Neural Networks in day-to-day, 
especially in business.
For this, we first proposed the development of a prototype simulator Artificial 
Neural Networks as web application, named NeuroGrubi, so people can not only 
use it, but also have access to the code. The Web was chosen because the easy 
access and availability offered to potential users of such systems.
The work will also indicate how this prototype can then be optimized and 
improved in the future to promote the use of techniques of Neural Networks. 
The purpose is to increase the use of neural networks technology by enterprises, 
leaving transparent to the end user, without specific details about the area.}

\keywords{Artificial Neural Networks, Web Application, Javascript.}

\receivedate{}

\acceptdate{}

\begin{document}

\maketitle


\newpage

\input{Introduction}
\section{Related Work}

The technology of artificial neural networks has been widely used in recent 
years in several areas of application. The use of this technology is already 
widespread in many areas, where the great performance objectives, monitoring and 
systems integration activities are targeted and increasingly arise solutions 
that use neural networks in new areas that exceed the solutions obtained by 
conventional \cite{1327478}.

Examples of some areas where the use of artificial neural networks have achieved 
great success:

\begin{itemize}
\item Water Flow Forecasting \cite{1556318}
\item Prediction of Electricity demand \cite{4679182}
\item Character Recognition \cite{5726693}
\item Stock Market Predictor \cite{5715226}
\item Signature Validation \cite{549040}
\item Characterizing and Evaluating Fraud in Electronic Transactions 
\cite{6392147}
\item Medical Image Diagnosis \cite{5953960}
\end{itemize}

For access to neural networks help a larger group of people, this work suggested 
the development of a web application that provides the user with an interesting 
solution to their problems.

It is apparent that a neural network has its computational power primarily 
related to its structure highly distributed and parallel and also their ability 
to learn and therefore generalize (ability of the neural network has to 
produce reasonable outputs for inputs not shown in phase learning/training). 
These two information-processing capabilities make it possible to solve complex 
problems through the use of techniques of Artificial Neural Networks. In 
practice, however, artificial neural networks can not by itself generate 
solutions. Should rather be integrated into a consistent approach to Systems 
Engineering. More specifically, one should take the complex problem which we 
wish to arrive at a solution, break it down into a number of relatively simple 
tasks and assign certain of these tasks (pattern recognition, associative 
memory, control, etc.). Techniques Artificial Neural Networks to solve them.

The use of Artificial Neural Networks offers the following useful properties and 
capabilities \cite{Bishop2005}:

\begin{enumerate}[I]
\item Non-Linearity.
\item Input-output Mapping.
\item Adaptation.
\item Contextual Information.
\item Failure tolerance.
\end{enumerate}
\section{NeuroGrubi}

Applications hosted on the Internet, or software as a service is an important 
issue nowadays. The main reasons to consider migrating your application to cloud 
projects are associated with reduced costs and increased output. Another 
important advantage is the ease of updating the software, requiring only update 
the application on the server and so all users can access the updates.

However, to be able to provide a service in which you must use technologies
work on the Internet. The NeuroGrubi was built using the language
Javascript programming, the NodeJS \cite{nodejs} as language interpreter on the 
server and MongoDB \cite{mongoDB}, one NoSQL database, used to store data for 
neural network training.

The library Brain.js \cite{brainjs} is a library with the Backpropagation 
algorithm that is used to carry out the training of the neural network. At the 
end of that library generates a function in javascript with the network already 
trained.

\subsection{Javascript e NodeJs}

The choice of programming language such as Javascript support NeuroGrubi is 
based facility that Javascript has to walk through objects. The objects in 
javascript behave like lists, thereby simplifying the training algorithm of the 
neural network. The algorithm rather than iterate over matrices, it iterate 
over objects list.

Another advantage in the choice of Javascript is its simple syntax of a loosely 
typed scripting language and dynamic typing. This facilitated the implementation 
of NeuroGrubi.

The NodeJS is a platform for developing applications in javascript. It provides 
an easy way to create scalable programs. As the user base grows NeuroGrubi, it 
is necessary to support more users and therefore will need to add more servers. 
The NodeJS allows this expansion simply.


\subsection{MongoDB}

MongoDB is a database document-oriented high-performance, open source, and 
schema-free, written in C++. He is a mix between repositories based scalable 
key / value and wealth of traditional features of relational databases.

These features help design NeuroGrubi. As the databases for training the neural 
network can be great, it makes the task of uploading the database arduous work. 
A solution is to upload only once in the database and use it in training as 
often as necessary.

The problem is that the training data is not of a generalization making the use 
of relational databases a complicated task. MongoDB makes the task of storing 
data in varying formats a simple task. MongoDB stores the entire line-shaped 
object, where each column represents an attribute.

\subsection{The Application}

First, as the NeuroGrubi works with private data the user, then must
provide a username and password. Thus the system can only provide
information the user entered in the system. The login screen is NeuroGrubi 
shown in figure \ref{fig:1}.

\begin{figure}[!htbp]
  \begin{center}
    \includegraphics[scale=0.24]{images/login}
    \caption{Login Screen}
    \label{fig:1}
  \end{center}
\end{figure}

The second screen is to view the databases that are already contained in 
NeuroGrubi. It also is possible upload others databases, as well as see the 
databases are already in NeuroGrubi. Another possible action on this screen is 
going to screen training Neural Network.

\begin{figure}[!htbp]
  \begin{center}
    \includegraphics[scale=0.18]{images/listaBD}
    \caption{Train Data Screen}
    \label{fig:2}
  \end{center}
\end{figure}

Third figure \ref{fig:3} shows how the data screen viewer appearance. 

\begin{figure}[!htbp]
  \begin{center}
    \includegraphics[scale=0.18]{images/dataViwer}
    \caption{Data Screen Viewer Appearance}
    \label{fig:3}
  \end{center}
\end{figure}

If you choose the train button, the screen directs you to NeuroGrubi training. 
On screen training the user has access configuration variables Backpropagation 
neural network used by the library Brais.JS. In this screen the User can 
configure the following parameters of the network.

\begin{itemize}
 \item Learning Rate
 \item Momentum
 \item Error Threshold
 \item Iteration
 \item Hidden Layers
 \item Node in each Hidden Layer
 \item Inputs and Output 
 \item Data for Test
\end{itemize}

Figure \ref{fig:4}  represents the configuration screen of the neural 
network training parameters.

\begin{figure}[!htbp]
  \begin{center}
    \includegraphics[scale=0.24]{images/confScreen}
    \caption{Configuration Training Screen}
    \label{fig:4}
  \end{center}
\end{figure}

After training the NeuroGrubi shows a screen with the result of training as 
shown on figure \ref{fig:5}

\begin{figure}[!htbp]
  \begin{center}
    \includegraphics[scale=0.17]{images/resultscreen}
    \caption{Train Result Screen}
    \label{fig:5}
  \end{center}
\end{figure}

\section{Conclusion and Future Work}

The NeuroGrubi was the first step towards to develop a web tool for 
neural networks. It certainly proved to be feasible by enabling people and 
organizations to have access a simple and free neural networks tool.

The research also shows that the technology of artificial neural networks can 
easily reach users not experts, since the developers have in mind what kind of 
user applications are intended to facilitate the use, leaving the fewest 
accessible possible to the configuration parameters. The NeuroGrubi was 
implemented with the goal of being easy to  non-experts users.

It is known that the arise of a new idea, as well as research on its 
feasibility and first steps for its implementation are quite difficult and 
complicated. It is believed that this step was completed this work, however, 
there is no doubt that there is much more to be done: starting with the 
implementation and analysis of other models of Artificial Neural Networks. 
Perhaps the development of a script specifying the steps to be carried out was 
of vital importance . Another possible work is changing the server-side 
technology, because \em{nodejs} and \em{javascript} are easy handling but are 
shown not to have a good performance when the script to train was running. 
Changing the programming language to \em{C++} or \em{Java} can improve the use 
of computer resources .

It is also essential to develop other applications more elaborate and refined. 
Could be carried out a research on various companies to find out which areas 
of Artificial Neural Networks techniques could be applied to improve 
performance and/or lower costs. Thus, the study could generate, with low cost, 
easily accepted tools and understanding to those potentially interested in 
a future purchase.

It is true that in one way or another, most applications and all the attention 
will be focused here on to the Internet. Offering these prototypes as web 
application will not only facilitate the exchange of idea on the part of 
developers, researchers and students as well. The internet makes the NeuroGrubi 
easier to access for people interested in using and learning Artificial Neural 
Network.




\bibliography{example}
\end{document}