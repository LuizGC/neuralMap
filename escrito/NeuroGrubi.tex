\section{NeuroGrubi}

Applications hosted on the Internet, or software as a service is an important 
issue nowadays. The main reasons to consider migrating your application to cloud 
projects are associated with reduced costs and increased output. Another 
important advantage is the ease of updating the software, requiring only update 
the application on the server and so all users can access the updates.

However, to be able to provide a service in which you must use technologies
work on the Internet. The NeuroGrubi was built using the language
Javascript programming, the NodeJS \cite{nodejs} as language interpreter on the 
server and MongoDB \cite{mongoDB}, one NoSQL database, used to store data for 
neural network training.

The library Brain.js \cite{brainjs} is a library with the Backpropagation 
algorithm that is used to carry out the training of the neural network. At the 
end of that library generates a function in javascript with the network already 
trained.

\subsection{Javascript e NodeJs}

The choice of programming language such as Javascript support NeuroGrubi is 
based facility that Javascript has to walk through objects. The objects in 
javascript behave like lists, thereby simplifying the training algorithm of the 
neural network. The algorithm rather than iterate over matrices, it iterate 
over objects list.

Another advantage in the choice of Javascript is its simple syntax of a loosely 
typed scripting language and dynamic typing. This facilitated the implementation 
of NeuroGrubi.

The NodeJS is a platform for developing applications in javascript. It provides 
an easy way to create scalable programs. As the user base grows NeuroGrubi, it 
is necessary to support more users and therefore will need to add more servers. 
The NodeJS allows this expansion simply.


\subsection{MongoDB}

MongoDB is a database document-oriented high-performance, open source, and 
schema-free, written in C++. He is a mix between repositories based scalable 
key / value and wealth of traditional features of relational databases.

These features help design NeuroGrubi. As the databases for training the neural 
network can be great, it makes the task of uploading the database arduous work. 
A solution is to upload only once in the database and use it in training as 
often as necessary.

The problem is that the training data is not of a generalization making the use 
of relational databases a complicated task. MongoDB makes the task of storing 
data in varying formats a simple task. MongoDB stores the entire line-shaped 
object, where each column represents an attribute.

\subsection{The Application}

First, as the NeuroGrubi works with private data the user, then must
provide a username and password. Thus the system can only provide
information the user entered in the system. The login screen is NeuroGrubi 
shown in figure \ref{fig:1}.

\begin{figure}[!htbp]
  \begin{center}
    \includegraphics[scale=0.24]{images/login}
    \caption{Login Screen}
    \label{fig:1}
  \end{center}
\end{figure}

The second screen is to view the databases that are already contained in 
NeuroGrubi. It also is possible upload others databases, as well as see the 
databases are already in NeuroGrubi. Another possible action on this screen is 
going to screen training Neural Network.

\begin{figure}[!htbp]
  \begin{center}
    \includegraphics[scale=0.18]{images/listaBD}
    \caption{Train Data Screen}
    \label{fig:2}
  \end{center}
\end{figure}

Third figure \ref{fig:3} shows how the data screen viewer appearance. 

\begin{figure}[!htbp]
  \begin{center}
    \includegraphics[scale=0.18]{images/dataViwer}
    \caption{Data Screen Viewer Appearance}
    \label{fig:3}
  \end{center}
\end{figure}

If you choose the train button, the screen directs you to NeuroGrubi training. 
On screen training the user has access configuration variables Backpropagation 
neural network used by the library Brais.JS. In this screen the User can 
configure the following parameters of the network.

\begin{itemize}
 \item Learning Rate
 \item Momentum
 \item Error Threshold
 \item Iteration
 \item Hidden Layers
 \item Node in each Hidden Layer
 \item Inputs and Output 
 \item Data for Test
\end{itemize}

Figure \ref{fig:4}  represents the configuration screen of the neural 
network training parameters.

\begin{figure}[!htbp]
  \begin{center}
    \includegraphics[scale=0.24]{images/confScreen}
    \caption{Configuration Training Screen}
    \label{fig:4}
  \end{center}
\end{figure}

After training the NeuroGrubi shows a screen with the result of training as 
shown on figure \ref{fig:5}

\begin{figure}[!htbp]
  \begin{center}
    \includegraphics[scale=0.17]{images/resultscreen}
    \caption{Train Result Screen}
    \label{fig:5}
  \end{center}
\end{figure}
